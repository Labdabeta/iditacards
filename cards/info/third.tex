\documentclass{iditacard}

\cardtype{utility}
\rarity{info}

\begin{document}
\begin{card}
\node [rectangle, minimum width=650.0/300.0, minimum height=950.0/300.0, text justified, text width=53mm, inner sep=1mm, anchor=north west] at (50.0/300.0,1000.0/300.0) {
    \fontsize{8}{8}

    \vskip -0.5em \hrule \vskip 0.5em
    {\large \textbf{Risk} \hfill 3}
    \vskip 0.5em \hrule \vskip 0.5em

    \textsc{Whatever} the risk of a card is, that many cards must be played from
    the top of your deck on the subsequent turns successfully. Only then does the
    original card's effect take place.

    \textsc{Additional} risks stack. If at any point you cannot afford the cards
    being played as risk, the risk card fizzles and does nothing.

    \textsc{Any} cards used as payment for risk have their effects occur when
    they are paid off, even if the parent risk card fizzles.
};
\end{card}
\end{document}
